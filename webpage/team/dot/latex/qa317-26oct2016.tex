%% Change the name of the author of this
%% document from CSC317 Computer Networks to
%% your name.

%% Make a PDF file from this document by
%% typing: pdflatex qa317-24oct2016.tex

\documentclass[twoside]{article}
\title{Graded Exercise 3}
\author{Dot Carmichael}
\date{26 October 2016}

\usepackage{amsmath}
\usepackage[colorlinks=true]{hyperref}

\setlength{\parindent}{0pt}
\setlength{\parskip}{6pt}

\newenvironment{answer}
  {\vspace*{0.2cm} \rule{12cm}{0.04cm} \vspace*{0.2cm}}
  {\vspace*{0.2cm}}

\begin{document}
\maketitle

\begin{enumerate}
  \item Match the people with their achievements.

  \begin{answer}

  \begin{tabular}{l|l}
    \textbf{Person} & \textbf{Achievement} \\ \hline
    Marc Andreesen & Ethernet \\
    Tim Berners-Lee & packet switching, Internet's first node \\
    Vint Cerf \& Bob Kahn & routing algorithm \\    
    Len Kleinrock & TCP/IP \\
    Bob Metcalfe & Web browser \\
    Radia Perlman & World Wide Web
    \end{tabular}

    \end{answer}

  \item A shared communications channel has a bandwidth of 3 Mbps.
    Each subscriber requires a bandwidth of 150 kbps when
    using the channel.
    Each subscriber uses the channel only 10\% of the time.

    This means that the capacity of the channel is sufficient to
    allow simultaneous communications by $20$ subscribers.

    This means that the probability $p$ that any given subscriber is active at any given moment
    is $0.1$.

    The probability that exactly $n$ of $N$ subscribers are simultaneously communicating
    when the probability that any individual subscriber is communicating is $p$ is:

    \begin{align*}
      P_{p} (n \mid N ) & = \frac{N!}{n! (N - n)!} (p)^n (1.0 - p)^{N - n}
      \end{align*}

    The probability that exactly n of 120 subscribers are simultaneously communicating
    is\ldots

    \begin{align*}
      P_{0.1} (n \mid 120 ) & = \frac{120!}{n! (120 - n)!} (0.1)^n (1.0 - 0.1)^{120 - n}
      \end{align*}

    With the help of a \href{http://stattrek.com/online-calculator/binomial.aspx}{calculator}
    we can find the probability that $21$ or more
    subscribers are active at once.
  
    The online calculator displays input fields for ``Probability of success on a single trial,''
    ``Number of trials,'' and ``Number of successes (x).''
    Enter values of $0.1$, $120$, and $21$ in these fields
    Click on the calculate button.
    Look for the probability that $21$ or more subscribers are simultaneously active
    in the output field that is labeled ``Cumulative probability: $P(X \ge 21)$.''

    \begin{enumerate}
      \item What is the probability that at any given moment
        $21$ or more subscribers will be communicating?
      \item What happens in a packet switched network when subscribers
        request more bandwidth than the channel can provide.
      \item What can you infer from this result about an advantage
        of packet switched communications?
      \end{enumerate}

  \begin{answer}

  \begin{enumerate}
    \item p=0.00794119224839296
    \item When more bandwith is requested than can be provided, a bottleneck occurs, 
    	and the queueing delay for packets increases, possibly to the point where the 
	queue is filled and packets begin to be dropped.
    \item Given the rather low probablility of more subscribers being active simultaneously 
    	than the system can handle, packet switch communications in this example can
	provide each subscriber as much bandwith as they need most of the time.
    \end{enumerate}

    \end{answer}

  \item
  \begin{enumerate}
    \item How many bits are in an IPv4 address?

    \item If all bits in an IPv4 address were available for specifying
      addresses of different machines on the Internet, how many hosts
      could the Internet connect?

    \item The dotted decimal notation is a way of writing IPv4 addresses.
      
      Find the IPv4 address of \verb+www.eonsahead.com+ by typing:

      \verb+nslookup www.eonsahead.com+

      Express the IPv4 address of \verb+www.eonsahead.com+ in
      dotted decimal notation.?

    \item The dotted decimal notation presents an address as
      four decimal numbers, each in the range of $0$ to $255$,
      and separated from one another by periods.

      A number $x$ in the range of $0$ to $255$ can be expressed
      in hexadecimal notation with two hexadecimal digits.

    \begin{itemize}
      \item The first (most significant) digit is $x / 16$.
        The division is integer division.

      \item The second (least significant) digit is $x \bmod 16$.

      \item Here is an example: $77_{10} \equiv 4D$ because\ldots
      \begin{itemize}
        \item  $77 / 16 = 4$
        \item $77 \bmod 16 = 13$
        \item the hexadecimal digit that represents $13$ is $D$
        \end{itemize}

    \begin{tabular}{ll|ll}
      \textbf{decimal} & \textbf{hexadecimal} & \textbf{decimial} & \textbf{hexadecmal} \\ \hline
      0 & 0 & 8 & 8 \\
      1 & 1 & 9 & 9 \\
      2 & 2 & 10 & A \\
      3 & 3 & 11 & B \\
      4 & 4 & 12 & C \\
      5 & 5 & 13 & D \\
      6 & 6 & 14 & E \\
      7 & 7 & 15 & F
      \end{tabular}

      \end{itemize}

      Express the address of \verb+www.eonsahead.com+ in hexadecimal
      notation.

    \item To translate a number from hexadecimal format to binary,
      replace each hexadecimal digit with the corresponding
      four bits bound in this table:

      \begin{tabular}{l|l}
        \textbf{Hexadecimal digit} & \textbf{Binary equivalent} \\ \hline
        0 & 0000 \\
        1 & 0001 \\
        2 & 0010 \\
        3 & 0011 \\
        4 & 0100 \\
        5 & 0101 \\
        6 & 0110 \\
        7 & 0111 \\
        8 & 1000 \\
        9 & 1001 \\
        A & 1010 \\
        B & 1011 \\
        C & 1100 \\
        D & 1101 \\
        E & 1110 \\
        F & 1111 \\
        \end{tabular}

      Express the IPv4 address of \verb+www.eonsahead.com+ in binary
      notation.
 
    \end{enumerate}

  \begin{answer}

  \begin{enumerate}
    \item There are 32 bits in an IPv4 address, divided into 4 8 byte segments.
    \item If every bit was available, there could be a total of 4,294,967,296
      IP addresses.
    \item The IP address of \verb+www.eonsahead.com+ is 45.40.136.115
    \item \verb+www.eonsahead.com+'s hexadecimal address is 2D.28.88.73
    \item \verb+www.eonsahead.com+'s binary address is 00101101.00101000.10001000.01110011
    \end{enumerate}

    \end{answer}

  \item An IPv6 address contains $128$ bits. With $128$ bits, it
    is possible to represent $2^{128}$ addresses.
  \begin{enumerate}
    \item Look up the current population of the earth. What integer
      power of two most closely approximates this number?
    \item What is the integer power of ten that most closely
      approximates $2^{128}$?

      Here are some relationships that you might find helpful.

      \begin{align*}
        2^{10} & \approx 1000 = 10^3 \\
        2^{20} & = 2^{10} \cdot 2^{10} \approx 10^3 \cdot 10^3 = 10^6 \\
        2^{30} & = 2^{10} \cdot 2^{10} \cdot 2^{10} \approx 10^3 \cdot 10^3 \cdot 10^3 = 10^9 \\
        2^{40} & = 2^{10} \cdot 2^{10} \cdot 2^{10} \cdot 2^{10} \approx
            10^3 \cdot 10^3 \cdot 10^3 \cdot 10^3 = 10^{12}
        \end{align*}

      In general, if $2^x = 10^y$, then $y$ is approximately equal to $x/y$: $2^{40} \approx 10^{40/3}$.
      
    \item How many IPv6 addresses would each person on earth have if the addresses
      were evenly distributed?
    \end{enumerate}

  \begin{answer}

  \begin{enumerate}
    \item The current population of the earth is between $2^{32}$ and $2^{33}$ people.
    \item $2^{128}$ is around three hundred and forty undecillion addresses.
    \item Every person would be able to have a very large number of addresses,
      approximately $47$ octillion apiece.
    \end{enumerate}

    \end{answer}

  \item Hundreds of millions of IPv4 addresses are reserved.
  \begin{enumerate}
    \item For what purposes are the CIDR address blocks 10.0.0.0/8 and 192.168.0.0/16 reserved?
    \item How many addresses are in the 10.0.0.0/8 block?
    \item How many addresses are in the 192.168.0.0/16?
    \item Use the \verb+ifconfig+ command to determine the IP address of
      one of the Linux machines in our laboratory. What is the address of the computer?
    \end{enumerate}

  \begin{answer}

  \begin{enumerate}
    \item 10.0.0.0/8 and 192.168.0.0/16 are reserved for private networks.
    \item There are 16,777,216 addresses in the 10.0.0.0/8 block.
    \item There are 65,536 addresses in the 192.168.0.0/16 block.
    \item The address of a computer in the lab is 10.101.6.3
    \end{enumerate}

    \end{answer}

  \item Port numbers are $16$ bit addresses. A $16$ bit address is large
    enough to specify any one of $65536$ different ports. What is a programmer
    addressing when a programmer specifies a port number?

  \begin{answer}
    A port number defines which network application a segment is being delivered to, as an 
    added layer of specificity over the IP address.
    \end{answer}

  \item Here is an address: d4:9a:20:0a:49:00
  \begin{enumerate}
    \item The format of this address matches the format of what kind of address?
    \item What does the address identify?
    \item The address contains 12 hexadecimal digits. How many bits are needed
      to specify this kind of address?
    \end{enumerate}
  
  \begin{answer}

  \begin{enumerate}
    \item This address is formatted as a MAC address.
    \item A MAC address is a link-layer address which uniquely identifies a link-layer 
    	adapter.
    \item 48 bits are needed for a 12 digit hexadecimal number.
    \end{enumerate}

    \end{answer}

  \item What kinds of addresses to the headers in each of these kinds
      of packets contain? (Your choices are IP addresses, MAC addresses,
      and port numbers.)
  \begin{enumerate}
    \item Transport layer segments
    \item Network layer datagrams
    \item Link layer Ethernet frames
    \end{enumerate}

  \begin{answer}

  \begin{enumerate}
    \item IP Address
    \item Port Number
    \item MAC Address
    \end{enumerate}

    \end{answer}

  \item DHCP is a client-server protocol.
  \begin{enumerate}
    \item What does this service provide to clients? (It may provide more than one item.) 
    \item Under what circumstances is a DHCP service especially useful?
    \item Which transport level protocol and which port number is used by
      a host to discover a DHCP server?
    \item DHCP uses an IP address that is reserved for broadcast. 
      A host that joins a network is a client.
      The exchange between a DHCP client and a DHCP server begins when
      the client broadcasts a \emph{discover message}.
      The client broadcasts this message because it does not yet know
      any addresses (including most particularly the address of the DHCP
      server) in the network.
      The server responds by broadcasting an \emph{offer message}.

      Why does the server also broadcast the \emph{offer message}?

    \end{enumerate}

  \begin{answer}

  \begin{enumerate}
    \item DHCP assigns an IP address to a host and provides its subnet mask, the address 
    	of the first router packets will visit upon leaving the host, and the address of 
	the local DNS server.
    \item DHCP is particularly useful for networks where users are constantly connecting and 
    	disconnecting and do not need to keep the same IP address for an extended period 
	of time.
    \item DHCP discovery messages are sent by UDP to port 67.
    \item The server broadcasts the offer message because it is possible for the request to 
    	reach and be returned by multiple DHCP servers. Because a client is not guaranteed 
	to choose that particular DHCP server, it waits to open the connection and send an 
	ACK until after the offer message gets a response.
    \end{enumerate}

    \end{answer}

  \item What are several possible objections to the use of NAT?

  \begin{answer}

  \begin{enumerate}
    \item Port numbers are meant to address processes, not hosts.
    \item Routers are only supposed to process packets up to layer 3.
    \item NAT violates the idea that hosts should talk directly to each other.
    \item If we switched to IPv6, NAT would be unnecessary.
    \end{enumerate}

    \end{answer}

  \item DNS and ARP both translate between two kinds of addresses.
  \begin{enumerate}
    \item Between which two kinds of addresses does DNS translate?
    \item Between which two kinds of addresses does ARP translate?
    \end{enumerate}

  \begin{answer}

  \begin{enumerate}
    \item DNS translates between IP addresses and hostnames.
    \item ARP translates between IP addresses and MAC addresses.
    \end{enumerate}

    \end{answer}

  \item 
  \begin{enumerate}
    \item The MTU (Maximum Transmission Unit) defines a limit at which
      layer of the Internet protocol stack?
    \item The MSS (Maximum Segment Size) defines a limit at which
      layer of the Internet protocol stack?
    \end{enumerate}

  \begin{answer}

  \begin{enumerate}
    \item Link Layer
    \item Transport Layer
    \end{enumerate}

    \end{answer}

  \item The link-state algorithm associates a number and a label with
    each node in a network. The algorithm assigns an initial value
    of $\infty$ to the number and $NULL$ (unknown) to the label.
    It updates these values in the course of its execution.
  \begin{enumerate}
    \item What does the number denote?
    \item What does the label denote?
    \end{enumerate}
  
  \begin{answer}

  \begin{enumerate}
    \item The number denotes the cost of the least-cost path to reach the node
    \item The label denotes the node which will be passed through directly 
    	before reaching the destination node on the least-cost path from the 
	source.
    \end{enumerate}

    \end{answer}

  \item The Bellman-Ford equation describes a relationship that
    is the basis of the distance-vector routing algorithm.

    Combine the following mathematical expressions
    (found in the first column of the table) to produce the Bellman-Ford equation.

  \begin{tabular}{ll}
    \textbf{expression} & \textbf{meaning} \\ \hline
    $d_x(y)$ & distance from $x$ to $y$ along shortest path between the two nodes\\
    $c(x,v)$ & length of edge that connects node $x$ to node $v$ \\
    $d_v(y)$ & distance from $v$ to $y$ along shortest path between the two nodes \\
    $\{ \cdots \}$ & a set \\
    ${\min}_v$ & \parbox[t]{10cm}{minimum value in collection of values that
      depend upon $v$, for all values of $v$}
    \end{tabular}

  \begin{answer}

  \begin{align*}
    \parbox{1cm}{$d_x(y)$} & = \parbox{4cm}{${\min}_v$\{$c(x,v)$ + $d_v(y)$\}}
    \end{align*}

    \end{answer}

  \item The distance-vector routing algorithm is\ldots
  \begin{enumerate}
    \item iterative or recursive?
    \item asynchronous or synchronous?
    \item centralized or distributed?
    \item self-terminating or terminated by a special signal?
    \end{enumerate}

  \begin{answer}

  \begin{enumerate}
    \item Iterative
    \item Asynchronous
    \item Distributed
    \item Self-terminating
    \end{enumerate}

    \end{answer}

  \item The Border Gateway Protocol (BGP) facilitates inter-AS routing.
  \begin{enumerate}
    \item What is inter-AS routing?
    \item Is scalability a more important concern in intra-AS routing
      or in inter-AS routing?
    \item Is performance (for example, the selection of the shortest
      route) a more important concern in intra-AS routing or
      in inter-AS routing?
    \end{enumerate}

  \begin{answer}

  \begin{enumerate}
    \item Inter-AS routing provides each autonomous system with a means 
	to obtain subnet reachability from neighboring ASs, distriubte 
	that information to all routers within the AS, and determine 
	"good" routes to subnets based on the information and policies.
    \item Scalability is more important in inter-AS routing.
    \item Performance is more important in intra-AS routing.
    \end{enumerate}

    \end{answer}

  \item In the following table, the first four elements of the
    first four rows are data. The fifth element in each of the 
    first four rows is a parity bit computed for that row.
    The elements of the fifth row are parity bits computed
    for the columns.

    The parity bits have been chosen to make the number of
    ones in a row (or column) even.
    For example, only one of the data bits in the first row
    is a one. Making the parity bit a one makes the total
    number of ones in the row even.

  \begin{tabular}{llll|l}
    0 & 0 & 0 & 1 & 1 \\
    1 & 0 & 0 & 1 & 0 \\
    0 & 1 & 1 & 0 & 0 \\
    0 & 1 & 0 & 0 & 1 \\ \hline
    1 & 0 & 1 & 0 & 0
    \end{tabular}

    Now suppose that noise in the communication line
    alters a bit during the transmission of this table.
    The receiver sees this table:

  \begin{tabular}{llll|l}
    0 & 0 & 0 & 1 & 1 \\
    1 & 0 & 0 & 1 & 0 \\
    0 & 1 & 0 & 0 & 0 \\
    0 & 1 & 0 & 0 & 1 \\ \hline
    1 & 0 & 1 & 0 & 0
    \end{tabular}

    \begin{enumerate}
      \item What can the receiver know?
      \item What can the receiver do?
      \end{enumerate}

  \begin{answer}

  \begin{enumerate}
    \item The receiver can know that there is an error.
    \item The receiver can tell which bit the error is in and correct the error.
    \end{enumerate}

    \end{answer}

  \item This problem is about a CSMA/CD network.
  \begin{itemize}
    \item Let $d_{prog}$ be the upper bound on the time required
      for front edge of a signal to travel (propagate) between two network adapters (interfaces).
    \item Let $d_{tran}$ be the time required to transmit the largest frame.
    \item Let $E$ be the efficiency of the network. ``Efficiency'' means the
      fraction of time (computed over a long run) in which frames are passing
      between adapters without collision. Assume that there are many adapters
      with many frames to send.
    \end{itemize}

  \begin{align*}
    E & \approx \frac{1}{1 + 5 \cdot \frac{d_{prog}}{d_{tran}}}
    \end{align*}

    Here is an analogy: Imagine a large school. There are many rooms.
    A single hallway connects the rooms. At unpredictable times, groups of students in
    one room pass through the hall to another room. A group of art
    students might move from the studio to the library. Another group of
    students might later move from the chemistry laboratory to the gymnasium.
    After that, a third group of students might walk from their English class to
    the nurse's office.

    A rule forbids more than one group of students to be in the hall at once.
    If the chemistry students open the door and see the art students in the hall,
    they must duck back into their laboratory, check the hallway for traffic
    again at a later time, and proceed only when the see at that the hallway
    is empty.

    In this case, $d_{prog}$ is a measure of how fast the students walk,
    and $d_{tran}$ is a measure of how much time it takes to get a group
    of students through a classroom door into the hall.
    The amount of time that a group spends in the hall is a function
    both of how fast the students walk and how many students are in the
    group.

    Efficiency is the fraction of time during which there are students
    in the hallway moving between rooms.
    
    \begin{enumerate}
      \item Does $E$ increase when $d_{prog}$ is increased or when it is decreased?
      \item Does $E$ increase when $d_{tran}$ is increased or when it is decreased?
      \end{enumerate}

  \begin{answer}

  \begin{enumerate}
    \item As $d_{prog}$ decreases, $E$ increases.
    \item As $d_{tran}$ increases, $E$ increases.
    \end{enumerate}

    \end{answer}

  \item Both routers and switchs store and forward packets.
    Routers execute layer 3 protocols. In layer 3, IP addresses determine
    where packets go.
    Switches execute layer 2 protocols. In layer 2, MAC addresses determine
    where packets go.
    Even so, network administrators can often choose to connect two networks
    using a router instead of a switch (or vice versa). 
    Network adminstrators need to know the advantages and disadvantages of
    each kind of device.

  \begin{enumerate}
    \item Hierarchical addressing and an easier avoidance of cycles (an attractive characteristic)
      is a property of which kind of device: router or switch?
    \item ``Plug-and-play'' (an attractive characteristic) is a property of which
      kind of device: router or switch?
    \item Faster processing of packets (an attractive characteristic) is a property
      of which kind of device: router or switch?
    \item The option of choosing from a greater variety of topologies (an attractive characteristic)
      follows from the selection of which kind of device: routers or switches?
      A greater variety of topologies means freedom from the constraint of avoiding multiple links
      between elements in a network. The network need not be a spanning tree, but can be a more
      general kind of graph.
    \item Susceptibilty to broadcast storms (an unattractive characteristic) is a property
      of which kind of device: router or switch?
    \end{enumerate}

  \begin{answer}

  \begin{enumerate}
    \item Router
    \item Switch
    \item Switch
    \item Router
    \item Switch
    \end{enumerate}

    \end{answer}

  \item Here is a computation for a cyclic redundancy check.

  \begin{align*}
    D & = 1010101010 \\
    G & = 10011
    \end{align*}

  \begin{tabular}{lllll|llllllllllllll}
      &   &   &   &   &   &   &   &   & 1 & 0 & 1 & 1 & 0 & 1 & 1 & 1 & 0 & 0    \\ \hline
    1 & 0 & 0 & 1 & 1 & 1 & 0 & 1 & 0 & 1 & 0 & 1 & 0 & 1 & 0 & 0 & 0 & 0 & 0  \\ 
      &   &   &   &   & 1 & 0 & 0 & 1 & 1 &   &   &   &   &   &   &   &   &    \\ \hline
      &   &   &   &   &   & 0 & 1 & 1 & 0 & 0 &   &   &   &   &   &   &   &    \\ 
      &   &   &   &   &   & 0 & 0 & 0 & 0 & 0 &   &   &   &   &   &   &   &    \\ \hline
      &   &   &   &   &   &   & 1 & 1 & 0 & 0 & 1 &   &   &   &   &   &   &    \\ 
      &   &   &   &   &   &   & 1 & 0 & 0 & 1 & 1 &   &   &   &   &   &   &    \\ \hline
      &   &   &   &   &   &   &   & 1 & 0 & 1 & 0 & 0 &   &   &   &   &   &    \\ 
      &   &   &   &   &   &   &   & 1 & 0 & 0 & 1 & 1 &   &   &   &   &   &    \\ \hline
      &   &   &   &   &   &   &   &   & 0 & 1 & 1 & 1 & 1 &   &   &   &   &    \\
      &   &   &   &   &   &   &   &   & 0 & 0 & 0 & 0 & 0 &   &   &   &   &    \\ \hline
      &   &   &   &   &   &   &   &   &   & 1 & 1 & 1 & 1 & 0 &   &   &   &    \\
      &   &   &   &   &   &   &   &   &   & 1 & 0 & 0 & 1 & 1 &   &   &   &    \\ \hline
      &   &   &   &   &   &   &   &   &   &   & 1 & 1 & 0 & 1 & 0 &   &   &    \\
      &   &   &   &   &   &   &   &   &   &   & 1 & 0 & 0 & 1 & 1 &   &   &    \\ \hline
      &   &   &   &   &   &   &   &   &   &   &   & 1 & 0 & 0 & 1 & 0 &   &    \\
      &   &   &   &   &   &   &   &   &   &   &   & 1 & 0 & 0 & 1 & 1 &   &    \\ \hline   
      &   &   &   &   &   &   &   &   &   &   &   & 0 & 0 & 0 & 0 & 1 &   &    \\ 
      &   &   &   &   &   &   &   &   &   &   &   &   & 0 & 0 & 0 & 0 & 0 &    \\ \hline
      &   &   &   &   &   &   &   &   &   &   &   &   &   & 0 & 0 & 1 & 0 & 0  \\
      &   &   &   &   &   &   &   &   &   &   &   &   &   & 0 & 0 & 0 & 0 & 0  \\ \hline  
      &   &   &   &   &   &   &   &   &   &   &   &   &   &   & 0 & 1 & 0 & 0  
    \end{tabular}

  \begin{align*}
    R & = 0100
    \end{align*}

  \begin{enumerate}
    \item The computation is a long division. As in any long division, the computation
      requires subtractions of multiples of the divisor. How does the method of subtraction
      differs from ordinary subtraction? 
    \item What is the bit string that the sender will transmit?
    \item What kinds of errors can the receiver detect?
    \end{enumerate}

  \begin{answer}

  \begin{enumerate}
    \item The subtraction has no borrows or carrys, and uses XOR instead of traditional 
	math operators.
    \item The sender transmits a number which is exactly divisible by a pre-determined 
	generator.
    \item The receiver can detect all burst errors up to the length of the 
	generator string, as well as most longer burst errors. In addition, any 
	odd number of errors can be detected no matter how many there are.
    \end{enumerate}

    \end{answer}

  \end{enumerate}

\end{document}
