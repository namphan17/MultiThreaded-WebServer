
%% Change the name of the author of this
%% document from CSC317 Computer Networks to
%% your name.

%% Make a PDF file from this document by
%% typing: pdflatex qa317-24oct2016.tex

\documentclass[twoside]{article}
\title{Graded Exercise 3}
\author{Ryan Conrardy}
\date{26 October 2016}

\usepackage{amsmath}
\usepackage[colorlinks=true]{hyperref}

\setlength{\parindent}{0pt}
\setlength{\parskip}{6pt}

\newenvironment{answer}
  {\vspace*{0.2cm} \rule{12cm}{0.04cm} \vspace*{0.2cm}}
  {\vspace*{0.2cm}}

\begin{document}
\maketitle

\begin{enumerate}
  \item Match the people with their achievements.

  \begin{tabular}{l|l}
    \textbf{Person} & \textbf{Achievement} \\ \hline
    Marc Andreesen & Ethernet \\
    Tim Berners-Lee & packet switching, Internet's first node \\
    Vint Cerf \& Bob Kahn & routing algorithm \\    
    Len Kleinrock & TCP/IP \\
    Bob Metcalfe & Web browser \\
    Radia Perlman & World Wide Web
    \end{tabular}

  \begin{answer}

  \begin{tabular}{l|l}
    \textbf{Person} & \textbf{Achievement} \\ \hline
    Marc Andreesen & Web browser\\
    Tim Berners-Lee &  World Wide Web \\
    Vint Cerf \& Bob Kahn & TCP/IP\\    
    Len Kleinrock & Packet switching, Internet's first node\\
    Bob Metcalfe & Ethernet \\
    Radia Perlman & Routing algorithm
    \end{tabular}

    \end{answer}

  \item A shared communications channel has a bandwidth of 3 Mbps.
    Each subscriber requires a bandwidth of 150 kbps when
    using the channel.
    Each subscriber uses the channel only 10\% of the time.

    This means that the capacity of the channel is sufficient to
    allow simultaneous communications by $20$ subscribers.

    This means that the probability $p$ that any given subscriber is active at any given moment
    is $0.1$.

    The probability that exactly $n$ of $N$ subscribers are simultaneously communicating
    when the probability that any individual subscriber is communicating is $p$ is:

    \begin{align*}
      P_{p} (n \mid N ) & = \frac{N!}{n! (N - n)!} (p)^n (1.0 - p)^{N - n}
      \end{align*}

    The probability that exactly n of 120 subscribers are simultaneously communicating
    is\ldots

    \begin{align*}
      P_{0.1} (n \mid 120 ) & = \frac{120!}{n! (120 - n)!} (0.1)^n (1.0 - 0.1)^{120 - n}
      \end{align*}

    With the help of a \href{http://stattrek.com/online-calculator/binomial.aspx}{calculator}
    we can find the probability that $21$ or more
    subscribers are active at once.
  
    The online calculator displays input fields for ``Probability of success on a single trial,''
    ``Number of trials,'' and ``Number of successes (x).''
    Enter values of $0.1$, $120$, and $21$ in these fields
    Click on the calculate button.
    Look for the probability that $21$ or more subscribers are simultaneously active
    in the output field that is labeled ``Cumulative probability: $P(X \ge 21)$.''

    \begin{enumerate}
      \item What is the probability that at any given moment
        $21$ or more subscribers will be communicating?
      \item What happens in a packet switched network when subscribers
        request more bandwidth than the channel can provide.
      \item What can you infer from this result about an advantage
        of packet switched communications?
      \end{enumerate}

  \begin{answer}

  \begin{enumerate}
    \item 0.00794119224839296
    \item The network becomes congested, causes packet delay, and it slows down transmission rates and causes queuing delays. Packets may be dropped in an extreamly crowded network due to either full buffers, or timeout errors. 
    \item Can allow multiple hosts to communicate over the network circuits unlike circuit switched networks, which are limited by the number of circuits.  Packet switching networks can also allocate bandwith better than circuit switched networks.
    \end{enumerate}

    \end{answer}

  \item
  \begin{enumerate}
    \item How many bits are in an IPv4 address?

    \item If all bits in an IPv4 address were available for specifying
      addresses of different machines on the Internet, how many hosts
      could the Internet connect?

    \item The dotted decimal notation is a way of writing IPv4 addresses.
      
      Find the IPv4 address of \verb+www.eonsahead.com+ by typing:

      \verb+nslookup www.eonsahead.com+

      Express the IPv4 address of \verb+www.eonsahead.com+ in
      dotted decimal notation.?

    \item The dotted decimal notation presents an address as
      four decimal numbers, each in the range of $0$ to $255$,
      and separated from one another by periods.

      A number $x$ in the range of $0$ to $255$ can be expressed
      in hexadecimal notation with two hexadecimal digits.

    \begin{itemize}
      \item The first (most significant) digit is $x / 16$.
        The division is integer division.

      \item The second (least significant) digit is $x \bmod 16$.

      \item Here is an example: $77_{10} \equiv 4D$ because\ldots
      \begin{itemize}
        \item  $77 / 16 = 4$
        \item $77 \bmod 16 = 13$
        \item the hexadecimal digit that represents $13$ is $D$
        \end{itemize}

    \begin{tabular}{ll|ll}
      \textbf{decimal} & \textbf{hexadecimal} & \textbf{decimial} & \textbf{hexadecmal} \\ \hline
      0 & 0 & 8 & 8 \\
      1 & 1 & 9 & 9 \\
      2 & 2 & 10 & A \\
      3 & 3 & 11 & B \\
      4 & 4 & 12 & C \\
      5 & 5 & 13 & D \\
      6 & 6 & 14 & E \\
      7 & 7 & 15 & F
      \end{tabular}

      \end{itemize}

      Express the address of \verb+www.eonsahead.com+ in hexadecimal
      notation.

    \item To translate a number from hexadecimal format to binary,
      replace each hexadecimal digit with the corresponding
      four bits bound in this table:

      \begin{tabular}{l|l}
        \textbf{Hexadecimal digit} & \textbf{Binary equivalent} \\ \hline
        0 & 0000 \\
        1 & 0001 \\
        2 & 0010 \\
        3 & 0011 \\
        4 & 0100 \\
        5 & 0101 \\
        6 & 0110 \\
        7 & 0111 \\
        8 & 1000 \\
        9 & 1001 \\
        A & 1010 \\
        B & 1011 \\
        C & 1100 \\
        D & 1101 \\
        E & 1110 \\
        F & 1111 \\
        \end{tabular}

      Express the IPv4 address of \verb+www.eonsahead.com+ in binary
      notation.
 
    \end{enumerate}

  \begin{answer}

  \begin{enumerate}
    \item There are 32 bits in an IPv4 addes, 4 bytes (8 bits) 4 times.
    \item $2^{32}$ possible addresses (4,294,967,296)
    \item 45.40.136.115
    \item 2D.28.88.73
    \item 00101101.00101000.10001000.01110011
    \end{enumerate}

    \end{answer}

  \item An IPv6 address contains 128 bits. With 128 bits, it
    is possible to represent $2^{128}$ addresses.
  \begin{enumerate}
    \item Look up the current population of the earth. What integer
      power of two most closely approximates this number?
    \item What is the integer power of ten that most closely
      approximates $2^{128}$?

      Here are some relationships that you might find helpful.

      \begin{align*}
        2^{10} & \approx 1000 = 10^3 \\
        2^{20} & = 2^{10} \cdot 2^{10} \approx 10^3 \cdot 10^3 = 10^6 \\
        2^{30} & = 2^{10} \cdot 2^{10} \cdot 2^{10} \approx 10^3 \cdot 10^3 \cdot 10^3 = 10^9 \\
        2^{40} & = 2^{10} \cdot 2^{10} \cdot 2^{10} \cdot 2^{10} \approx
            10^3 \cdot 10^3 \cdot 10^3 \cdot 10^3 = 10^{12}
        \end{align*}

      In general, if $2^x = 10^y$, then $y$ is approximately equal to $x/y$: $2^{40} \approx 10^{40/3}$.
      
    \item How many IPv6 addresses would each person on earth have if the addresses
      were evenly distributed?
    \end{enumerate}

  \begin{answer}

  \begin{enumerate}
    \item Current earth population: 7.125 billion (2013 census).  This is approx $2^{33}$ (8.589 billion).
    \item $2^{128} = 2^{10}$...$2^{10}$(12 times) =~= $10^{3}$...$10^{3}$(12 times) = $10^{38}$
    \item 4.7758929e+28 (a lot). (or about $5*10^{28}$ according to \href{http://itknowledgeexchange.techtarget.com/whatis/ipv6-addresses-how-many-is-that-in-numbers/}{Site}
    \end{enumerate}

    \end{answer}

  \item Hundreds of millions of IPv4 addresses are reserved.
  \begin{enumerate}
    \item For what purposes are the CIDR address blocks 10.0.0.0/8 and 192.168.0.0/16 reserved?
    \item How many addresses are in the 10.0.0.0/8 block?
    \item How many addresses are in the 192.168.0.0/16?
    \item Use the \verb+ifconfig+ command to determine the IP address of
      one of the Linux machines in our laboratory. What is the address of the computer?
    \end{enumerate}

  \begin{answer}

  \begin{enumerate}
    \item 10.0.0.0/8 block is reserved for local communications within a private network.  The 192.168.0.0/16 block is reserved for local communications within a private network.
    \item 10.0.0.0/8 has the range 10.0.0.0-10.255.255.255 (16,777,216 addresses).
    \item 192.168.00/16 has the range 192.168.0.0-192.88.99.255 (65,536 addresses).
    \item eno1 IP  inet address: 10.101.6.38
    \end{enumerate}

    \end{answer}

  \item Port numbers are $16$ bit addresses. A $16$ bit address is large
    enough to specify any one of $65536$ different ports. What is a programmer
    addressing when a programmer specifies a port number?

  \begin{answer}
    A port is an endpoint of communication, specific port numbers generally identify specific services from a host such as webservers, Telnet, DNS, etc.  It is supposed to address processes.  Uses TCP and IP.  
    \end{answer}

  \item Here is an address: d4:9a:20:0a:49:00
  \begin{enumerate}
    \item The format of this address matches the format of what kind of address?
    \item What does the address identify?
    \item The address contains 12 hexadecimal digits. How many bits are needed
      to specify this kind of address?
    \end{enumerate}
  
  \begin{answer}

  \begin{enumerate}
    \item MAC address
    \item Harware/physical address of the network adaptor device.  Should be globally unique.
    \item 48 bits.  (6-bytes, 6 1-byte 2-hexidecimal numbers). 
    \end{enumerate}

    \end{answer}

  \item What kinds of addresses to the headers in each of these kinds
      of packets contain? (Your choices are IP addresses, MAC addresses,
      and port numbers.)
  \begin{enumerate}
    \item Transport layer segments
    \item Network layer datagrams
    \item Link layer Ethernet frames
    \end{enumerate}

  \begin{answer}

  \begin{enumerate}
    \item Port
    \item IP address
    \item MAC address
    \end{enumerate}

    \end{answer}

  \item DHCP is a client-server protocol.
  \begin{enumerate}
    \item What does this service provide to clients? (It may provide more than one item.) 
    \item Under what circumstances is a DHCP service especially useful?
    \item Which transport level protocol and which port number is used by
      a host to discover a DHCP server?
    \item DHCP uses an IP address that is reserved for broadcast. 
      A host that joins a network is a client.
      The exchange between a DHCP client and a DHCP server begins when
      the client broadcasts a \emph{discover message}.
      The client broadcasts this message because it does not yet know
      any addresses (including most particularly the address of the DHCP
      server) in the network.
      The server responds by broadcasting an \emph{offer message}.

      Why does the server also broadcast the \emph{offer message}?

    \end{enumerate}

  \begin{answer}

  \begin{enumerate}
    \item Dynamic Host Configuration Protocol provides an Internet Protocol host with it's IP address automatically and other information such as the subnet mask and default gateway of it's local network.  Also may give address for DNS servers and DNS domain name.  
    \item Especially useful when connecting a computer to a network for the first time and you don't want to manually create an IP address.  And when hosts come and go into a network often.
    \item It is implemented with two UDP port numbers; UDP port number 67 is the destination port of a server, and UDP port number 68 is used by the client.
    \item Because the client does not yet have an IP address, thus there is no way to directly send a message to it.
    \end{enumerate}

    \end{answer}

  \item What are several possible objections to the use of NAT?

  \begin{answer}

  \begin{enumerate}
    \item Port numbers are meant to be used for addressing processes, not addressing hosts
    \item Routers are supposed to process packets only up to layer 3
    \item NAT protocol violates the end-to-end argument
    \item We should use IPv6 to solve the shortage of IP addresses
    \end{enumerate}

    \end{answer}

  \item DNS and ARP both translate between two kinds of addresses.
  \begin{enumerate}
    \item Between which two kinds of addresses does DNS translate?
    \item Between which two kinds of addresses does ARP translate?
    \end{enumerate}

  \begin{answer}

  \begin{enumerate}
    \item DNS translates between Host/Domain Name and IP addresses. (or vice versa)
    \item Hardware address (MAC address) and Network layer address (IP address).
    \end{enumerate}

    \end{answer}

  \item 
  \begin{enumerate}
    \item The MTU (Maximum Transmission Unit) defines a limit at which
      layer of the Internet protocol stack?
    \item The MSS (Maximum Segment Size) defines a limit at which
      layer of the Internet protocol stack?
    \end{enumerate}

  \begin{answer}

  \begin{enumerate}
    \item Network layer
    \item Transport layer (TCP and IP)
    \end{enumerate}

    \end{answer}

  \item The link-state algorithm associates a number and a label with
    each node in a network. The algorithm assigns an initial value
    of $\infty$ to the number and $NULL$ (unknown) to the label.
    It updates these values in the course of its execution.
  \begin{enumerate}
    \item What does the number denote?
    \item What does the label denote?
    \end{enumerate}
  
  \begin{answer}

  \begin{enumerate}
    \item The number is the 'cost' on the least cost path from the source to get to the node.
    \item The label is the node which is the previous node on the least cost path from the source node.
    \end{enumerate}

    \end{answer}

  \item The Bellman-Ford equation describes a relationship that
    is the basis of the distance-vector routing algorithm.

    Combine the following mathematical expressions
    (found in the first column of the table) to produce the Bellman-Ford equation.

  \begin{tabular}{ll}
    \textbf{expression} & \textbf{meaning} \\ \hline
    $d_x(y)$ & distance from $x$ to $y$ along shortest path between the two nodes\\
    $c(x,v)$ & length of edge that connects node $x$ to node $v$ \\
    $d_v(y)$ & distance from $v$ to $y$ along shortest path between the two nodes \\
    $\{ \cdots \}$ & a set \\
    ${\min}_v$ & \parbox[t]{10cm}{minimum value in collection of values that
      depend upon $v$, for all values of $v$}
    \end{tabular}

  \begin{answer}

  \begin{align*}
    d_x(y) = (\min)_v\{c(x,v)+d_v(y)\}
    \end{align*}

    \end{answer}

  \item The distance-vector routing algorithm is\ldots
  \begin{enumerate}
    \item iterative or recursive?
    \item asynchronous or synchronous?
    \item centralized or distributed?
    \item self-terminating or terminated by a special signal?
    \end{enumerate}

  \begin{answer}

  \begin{enumerate}
    \item Iterative
    \item Asynchronous
    \item Distributed
    \item Self-terminating
    \end{enumerate}

    \end{answer}

  \item The Border Gateway Protocol (BGP) facilitates inter-AS routing.
  \begin{enumerate}
    \item What is inter-AS routing?
    \item Is scalability a more important concern in intra-AS routing
      or in inter-AS routing?
    \item Is performance (for example, the selection of the shortest
      route) a more important concern in intra-AS routing or
      in inter-AS routing?
    \end{enumerate}

  \begin{answer}

  \begin{enumerate}
    \item Communicating between two or more ASs (autonomous systems), ability to obtain reachability nformation from neighboring ASs and propagating the reachability information to all routers internal to the AS 
    \item Inter-AS as there are tons of routers in the internet, especially between the ASs**
    \item Intra-AS
    \end{enumerate}

    \end{answer}

  \item In the following table, the first four elements of the
    first four rows are data. The fifth element in each of the 
    first four rows is a parity bit computed for that row.
    The elements of the fifth row are parity bits computed
    for the columns.

    The parity bits have been chosen to make the number of
    ones in a row (or column) even.
    For example, only one of the data bits in the first row
    is a one. Making the parity bit a one makes the total
    number of ones in the row even.

  \begin{tabular}{llll|l}
    0 & 0 & 0 & 1 & 1 \\
    1 & 0 & 0 & 1 & 0 \\
    0 & 1 & 1 & 0 & 0 \\
    0 & 1 & 0 & 0 & 1 \\ \hline
    1 & 0 & 1 & 0 & 0
    \end{tabular}

    Now suppose that noise in the communication line
    alters a bit during the transmission of this table.
    The receiver sees this table:

  \begin{tabular}{llll|l}
    0 & 0 & 0 & 1 & 1 \\
    1 & 0 & 0 & 1 & 0 \\
    0 & 1 & 0 & 0 & 0 \\
    0 & 1 & 0 & 0 & 1 \\ \hline
    1 & 0 & 1 & 0 & 0
    \end{tabular}

    \begin{enumerate}
      \item What can the receiver know?
      \item What can the receiver do?
      \end{enumerate}

  \begin{answer}

  \begin{enumerate}
    \item That collum 3 and row 3 are incorrectly labled.  This is because the pairty bit for collum 3 is 1, but there are no 1s; and the parity bit for row 3 is 0, but there is one 1.
    \item Since the receiver knows that bit misplaced must be in both row 3 and collum 3 then it can change the 0 to a 1.
    \end{enumerate}

    \end{answer}

  \item This problem is about a CSMA/CD network.
  \begin{itemize}
    \item Let $d_{prog}$ be the upper bound on the time required
      for front edge of a signal to travel (propagate) between two network adapters (interfaces).
    \item Let $d_{tran}$ be the time required to transmit the largest frame.
    \item Let $E$ be the efficiency of the network. ``Efficiency'' means the
      fraction of time (computed over a long run) in which frames are passing
      between adapters without collision. Assume that there are many adapters
      with many frames to send.
    \end{itemize}

  \begin{align*}
    E & \approx \frac{1}{1 + 5 \cdot \frac{d_{prog}}{d_{tran}}}
    \end{align*}

    Here is an analogy: Imagine a large school. There are many rooms.
    A single hallway connects the rooms. At unpredictable times, groups of students in
    one room pass through the hall to another room. A group of art
    students might move from the studio to the library. Another group of
    students might later move from the chemistry laboratory to the gymnasium.
    After that, a third group of students might walk from their English class to
    the nurse's office.

    A rule forbids more than one group of students to be in the hall at once.
    If the chemistry students open the door and see the art students in the hall,
    they must duck back into their laboratory, check the hallway for traffic
    again at a later time, and proceed only when the see at that the hallway
    is empty.

    In this case, $d_{prog}$ is a measure of how fast the students walk,
    and $d_{tran}$ is a measure of how much time it takes to get a group
    of students through a classroom door into the hall.
    The amount of time that a group spends in the hall is a function
    both of how fast the students walk and how many students are in the
    group.

    Efficiency is the fraction of time during which there are students
    in the hallway moving between rooms.
    
    \begin{enumerate}
      \item Does $E$ increase when $d_{prog}$ is increased or when it is decreased?
      \item Does $E$ increase when $d_{tran}$ is increased or when it is decreased?
      \end{enumerate}

  \begin{answer}

  \begin{enumerate}
    \item $E$ increases when $d_{prog}$ Decreases
    \item $E$ increases when $d_{tran}$ Increases
    \end{enumerate}

    \end{answer}

  \item Both routers and switchs store and forward packets.
    Routers execute layer 3 protocols. In layer 3, IP addresses determine
    where packets go.
    Switches execute layer 2 protocols. In layer 2, MAC addresses determine
    where packets go.
    Even so, network administrators can often choose to connect two networks
    using a router instead of a switch (or vice versa). 
    Network adminstrators need to know the advantages and disadvantages of
    each kind of device.

  \begin{enumerate}
    \item Hierarchical addressing and an easier avoidance of cycles (an attractive characteristic)
      is a property of which kind of device: router or switch?
    \item ``Plug-and-play'' (an attractive characteristic) is a property of which
      kind of device: router or switch?
    \item Faster processing of packets (an attractive characteristic) is a property
      of which kind of device: router or switch?
    \item The option of choosing from a greater variety of topologies (an attractive characteristic)
      follows from the selection of which kind of device: routers or switches?
      A greater variety of topologies means freedom from the constraint of avoiding multiple links
      between elements in a network. The network need not be a spanning tree, but can be a more
      general kind of graph.
    \item Susceptibilty to broadcast storms (an unattractive characteristic) is a property
      of which kind of device: router or switch?
    \end{enumerate}

  \begin{answer}

  \begin{enumerate}
    \item Routers
    \item Switch
    \item Switch
    \item Router
    \item Switch
    \end{enumerate}

    \end{answer}

  \item Here is a computation for a cyclic redundancy check.

  \begin{align*}
    D & = 1010101010 \\
    G & = 10011
    \end{align*}

  \begin{tabular}{lllll|llllllllllllll}
      &   &   &   &   &   &   &   &   & 1 & 0 & 1 & 1 & 0 & 1 & 1 & 1 & 0 & 0    \\ \hline
    1 & 0 & 0 & 1 & 1 & 1 & 0 & 1 & 0 & 1 & 0 & 1 & 0 & 1 & 0 & 0 & 0 & 0 & 0  \\ 
      &   &   &   &   & 1 & 0 & 0 & 1 & 1 &   &   &   &   &   &   &   &   &    \\ \hline
      &   &   &   &   &   & 0 & 1 & 1 & 0 & 0 &   &   &   &   &   &   &   &    \\ 
      &   &   &   &   &   & 0 & 0 & 0 & 0 & 0 &   &   &   &   &   &   &   &    \\ \hline
      &   &   &   &   &   &   & 1 & 1 & 0 & 0 & 1 &   &   &   &   &   &   &    \\ 
      &   &   &   &   &   &   & 1 & 0 & 0 & 1 & 1 &   &   &   &   &   &   &    \\ \hline
      &   &   &   &   &   &   &   & 1 & 0 & 1 & 0 & 0 &   &   &   &   &   &    \\ 
      &   &   &   &   &   &   &   & 1 & 0 & 0 & 1 & 1 &   &   &   &   &   &    \\ \hline
      &   &   &   &   &   &   &   &   & 0 & 1 & 1 & 1 & 1 &   &   &   &   &    \\
      &   &   &   &   &   &   &   &   & 0 & 0 & 0 & 0 & 0 &   &   &   &   &    \\ \hline
      &   &   &   &   &   &   &   &   &   & 1 & 1 & 1 & 1 & 0 &   &   &   &    \\
      &   &   &   &   &   &   &   &   &   & 1 & 0 & 0 & 1 & 1 &   &   &   &    \\ \hline
      &   &   &   &   &   &   &   &   &   &   & 1 & 1 & 0 & 1 & 0 &   &   &    \\
      &   &   &   &   &   &   &   &   &   &   & 1 & 0 & 0 & 1 & 1 &   &   &    \\ \hline
      &   &   &   &   &   &   &   &   &   &   &   & 1 & 0 & 0 & 1 & 0 &   &    \\
      &   &   &   &   &   &   &   &   &   &   &   & 1 & 0 & 0 & 1 & 1 &   &    \\ \hline   
      &   &   &   &   &   &   &   &   &   &   &   & 0 & 0 & 0 & 0 & 1 &   &    \\ 
      &   &   &   &   &   &   &   &   &   &   &   &   & 0 & 0 & 0 & 0 & 0 &    \\ \hline
      &   &   &   &   &   &   &   &   &   &   &   &   &   & 0 & 0 & 1 & 0 & 0  \\
      &   &   &   &   &   &   &   &   &   &   &   &   &   & 0 & 0 & 0 & 0 & 0  \\ \hline  
      &   &   &   &   &   &   &   &   &   &   &   &   &   &   & 0 & 1 & 0 & 0  
    \end{tabular}

  \begin{align*}
    R & = 0100
    \end{align*}

  \begin{enumerate}
    \item The computation is a long division. As in any long division, the computation
      requires subtractions of multiples of the divisor. How does the method of subtraction
      differs from ordinary subtraction? 
    \item What is the bit string that the sender will transmit?
    \item What kinds of errors can the receiver detect?
    \end{enumerate}

  \begin{answer}

  \begin{enumerate}
    \item Uses XOR (exclusive Or) subtraction.  If it is ever two different bits the number is 1, but if it is two same bits the number is 0.
    \item Sends the D value (piece of data) 1010101010, and additional R bits.
    \item The Reviever detects bit errors including burst errors. Reciever divides the D+R by G and if it is a non-zero answer there are errors.  Thus it can detect burst errors of fewer than r+1 bits, and even so a burst error with r+1 bits is detected with probability of $1 - 0.5^{r}$.  Also it can detect any odd number of bit errors.
    \end{enumerate}

    \end{answer}

  \end{enumerate}

\end{document}
